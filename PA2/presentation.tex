\documentclass{article}
\usepackage[utf8]{inputenc}
\usepackage{xeCJK}
\usepackage{amsmath}
\begin{document}
\section{向量几何法}
对于任意的旋转轴$\omega$($\omega$是一个三维单位向量),和任意的向量$v$,
$v$总可以分解为$v_{\perp}$和$v_{//}$,其中
$v_{//}=(v \cdot \omega) v$,$v_{\perp}=v-v_{//}$。
在旋转过程中,平行于旋转轴的$v_{//}$不会发生变换,只有垂直于旋转轴的$v_{\perp}$会发生变换。
旋转后为:
$$v_{\perp}^{'}=cos(\theta)*v_{\perp}+sin(\theta)*v\times \omega$$
故
\begin{equation}
    \begin{split}
        v^{'}=&v_{//}+v_{\perp}^{'}\\
        =&(v \cdot \omega) v+cos(\theta)v_{\perp}+sin(\theta)v\times \omega\\
        =&(1-cos(\theta))(v\cdot\omega)\omega+cos(\theta)v+sin(\theta)(\omega \times v)\\
        % =&(1-cos(\theta))\omega \omega^{T} v+cos(\theta) v+sin(\theta)\omega^{^}v\\
        =&(1-cos\theta)\omega \omega^T v + cos\theta v + sin\theta \omega^{\^} v
        % =&((1-cos(\theta))\omega \omega^{T}+cos(\theta)+sin(\theta)\omega^{^})v\\
    \end{split}
\end{equation}
%%%%%%%%%%%%%%%%%%%%%%%%%%%%%%%%%%%%%%%%%%%55
\section{李代数法}
\end{document}